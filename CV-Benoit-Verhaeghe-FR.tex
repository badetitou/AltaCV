%%%%%%%%%%%%%%%%%
% AltaCV - Version FR (revu)
%%%%%%%%%%%%%%%%

% \documentclass[10pt,a4paper,normalphoto]{altacv}
\documentclass[10pt,a4paper,ragged2e,withhyper]{altacv}

\geometry{left=1.25cm,right=1.25cm,top=1.5cm,bottom=1.5cm,columnsep=1.2cm}

\usepackage{paracol}
\usepackage[utf8]{inputenc}
\usepackage[T1]{fontenc}
\usepackage[french]{babel}

\ifxetexorluatex
  \setmainfont{Roboto Slab}
  \setsansfont{Lato}
  \renewcommand{\familydefault}{\sfdefault}
\else
  \usepackage[rm]{roboto}
  \usepackage[defaultsans]{lato}
  \renewcommand{\familydefault}{\sfdefault}
\fi

\definecolor{SlateGrey}{HTML}{2E2E2E}
\definecolor{LightGrey}{HTML}{666666}
\definecolor{DarkPastelRed}{HTML}{450808}
\definecolor{PastelRed}{HTML}{8F0D0D}
\definecolor{GoldenEarth}{HTML}{E7D192}

\colorlet{name}{black}
\colorlet{tagline}{PastelRed}
\colorlet{heading}{DarkPastelRed}
\colorlet{headingrule}{GoldenEarth}
\colorlet{subheading}{PastelRed}
\colorlet{accent}{PastelRed}
\colorlet{emphasis}{SlateGrey}
\colorlet{body}{LightGrey}

\renewcommand{\namefont}{\Huge\rmfamily\bfseries}
\renewcommand{\personalinfofont}{\footnotesize}
\renewcommand{\cvsectionfont}{\LARGE\rmfamily\bfseries}
\renewcommand{\cvsubsectionfont}{\large\bfseries}

\renewcommand{\itemmarker}{{\small\textbullet}}
\renewcommand{\ratingmarker}{\faCircle}

% Publications
\input{pubs-num.cfg}
\addbibresource{sample.bib}

\newcommand*\leftright[3]{%
  \leavevmode
  \rlap{#2}%
  \hspace{#1}%
  #3}

% Toggle version longue
\newif\iflongcv
\longcvtrue % <-- mettre \longcvfalse pour n'afficher QUE le CV court

\begin{document}

\name{Benoit Verhaeghe, Ph.D.}
\tagline{Leader technique | IA pour le code | Modernisation logicielle | GreenIT / Numérique responsable}
\photoR{2.8cm}{profil}

\personalinfo{%
  \email{work@badetitou.fr}
  \phone{+33 6 58 33 53 74}
  % \mailaddress{4, rue de la Moselle} % Optionnel
  \location{Grenoble, France}
  \homepage{badetitou.fr}
  \linkedin{benoitverhaeghe}
  \github{badetitou}
  \orcid{0000-0002-4588-2698}
}

\makecvheader

\columnratio{0.62}
\begin{paracol}{2}

% =========================
% COLONNE GAUCHE
% =========================

\cvsection{Profil}

Docteur en génie logiciel, à l’interface entre \textbf{recherche appliquée} et \textbf{déploiement industriel}. Je pilote des initiatives autour de l’\textbf{IA pour le développement}, de la \textbf{modernisation de logiciels à grande échelle} (ex.\ \textbf{1M+ lignes de code}), et de la \textbf{mesure GreenIT}. Mainteneur open-source d’une extension VS Code (\textbf{4\,000+} installations).

\smallskip

\cvsection{Expérience}

\cvevent{Chef de projet scientifique / Responsable R\&D}{Berger-Levrault}{janv. 2024 -- présent}{Lyon, France}
\begin{itemize}
  \item Management d’une \textbf{équipe de 6} ingénieurs et chercheurs ; encadrement de stagiaires et doctorants.
  \item Définition de la feuille de route R\&D et coordination des collaborations avec \textbf{Inria} et les unités produit/industrie.
  \item Conception et pilotage d’études mixtes (qualitatives + quantitatives) pour mesurer l’impact sur la \textbf{productivité des développeurs}.
  \item Pilotage d’initiatives \textbf{IA pour le code} (IDE \& workflows de merge requests) : intégration MCP / tool-calling et mesure d’impact en conditions réelles.
  \item Lancement et pilotage de la mesure \textbf{GreenIT} (infrastructure + application/code) et accompagnement de la modernisation via la \textbf{cartographie logicielle}.
\end{itemize}

\divider

\cvevent{Ingénieur logiciel R\&D}{Berger-Levrault}{nov. 2021 -- déc. 2023}{Lyon, France}
\begin{itemize}
  \item Contribution à la modernisation de front-ends Java (GWT/Swing, \textbf{1M+ LOC}) vers \textbf{Angular}.
  \item Mise en place de stratégies de migration incrémentales (architectures hybrides) pour réduire les risques de delivery.
  \item Encadrement de doctorants et stagiaires ; co-publications ; transfert technologique et partenariats académiques.
\end{itemize}

\divider

\cvevent{Ingénieur logiciel doctorant (doctorat industriel)}{Inria RMoD / Berger-Levrault}{mars 2018 -- oct. 2021}{Montpellier \& Lille, France}
\begin{itemize}
  \item Conception d’outils de migration d’interfaces (GUI) de \textbf{GWT} vers \textbf{Angular} via \textbf{l’ingénierie dirigée par les modèles (IDM/MDE)}.
  \item Analyses d’architecture pour soutenir la planification et la modernisation.
  \item Enseignement et publications dans des conférences/journaux de génie logiciel.
\end{itemize}

\cvsection{Projets sélectionnés}

\cvevent{Extension VS Code --- Support du langage Pharo}{Open-source (projet personnel)}{déc. 2020 -- présent}{}
\begin{itemize}
  \item Expérience développeur Pharo complète dans VS Code : support langage, débogage et notebooks.
  \item Intégration de fonctionnalités d’assistance par IA pour les workflows Pharo.
  \item \textbf{4\,000+} installations sur la marketplace VS Code.
\end{itemize}

\divider

\cvevent{Casino --- Plateforme de migration GUI}{Inria RMoD / Berger-Levrault}{mars 2018 -- déc. 2023}{}
\begin{itemize}
  \item Migration automatique d’interfaces applicatives ; livraison de \textbf{6 importeurs} et \textbf{5 exporteurs}.
\end{itemize}

% =========================
% COLONNE DROITE
% =========================
\switchcolumn

\cvsection{Prix \& distinctions}

\cvachievement{\faTrophy}{Prix Science ouverte de la thèse}{Thèse récompensée par le \href{https://www.enseignementsup-recherche.gouv.fr/fr/remise-des-premiers-prix-science-ouverte-de-la-these-97810}{Prix Science ouverte de la thèse} (Ministère de l’Enseignement supérieur et de la Recherche).}
\divider

\cvachievement{\faTrophy}{Prix Innovation (ESUG 2022) --- 3e place}{\href{https://marketplace.visualstudio.com/items?itemName=badetitou.pharo-language-server}{Pharo Language Server} élu 3e au Prix Innovation ESUG 2022.}
\divider

\cvachievement{\faTrophy}{Best Paper Award}{Migrating GWT to Angular 6 using MDE @ Sattose.}
\divider

\cvachievement{\faTrophy}{Prix Innovation (ESUG 2017) --- 2e place}{\href{https://badetitou.fr/projects/SmartTest/}{SmartTest} élu 2e au Prix Innovation ESUG 2017.}

\cvsection{Compétences}

\textbf{Domaines clés}\par
\cvtag{Architecture logicielle}
\cvtag{Modernisation}
\cvtag{Reverse engineering}
\cvtag{IDM/MDE}\par
\divider

\textbf{IA \& mesure}\par
\cvtag{IA pour le code}
\cvtag{Productivité dev}
\cvtag{Évaluation d’outils}
\cvtag{GreenIT}\par
\divider

\textbf{Langages \& outils}\par
\cvtag{Pharo}
\cvtag{Java}
\cvtag{Angular}
\cvtag{TypeScript}
\cvtag{REST}\par
\cvtag{Git}
\cvtag{Jenkins}
\cvtag{GitHub/GitLab CI/CD}\par
\divider

\textbf{Leadership}\par
\cvtag{Pilotage projet}
\cvtag{Collaboration}
\cvtag{Mentorat}
\cvtag{Communication}

\cvsection{Langues}

\leftright{5em}{\textbf{Français}}{Langue maternelle}

\divider

\leftright{5em}{\textbf{Anglais}}{Courant}

\medskip

\cvsection{Formation}

\cvevent{Doctorat en génie logiciel}{Université de Lille / Berger-Levrault}{janv. 2019 -- oct. 2021}{}
Approche incrémentale de migration d’IHM applicatives via des métamodèles

\divider

\cvevent{Diplôme d’ingénieur (Génie logiciel \& Statistiques)}{Polytech Lille}{sept. 2015 -- juil. 2018}{}

\divider

\cvevent{DUT Informatique}{IUT A -- Université de Lille}{sept. 2014 -- juin 2015}{}

\cvsection{Centres d’intérêt}
\cvtag{Jeux de société}
\cvtag{Open-source}

\end{paracol}

% =========================
% VERSION DÉTAILLÉE (optionnelle)
% =========================
\iflongcv
\newpage

\cvsection{CV détaillé}

\cvsection{Expérience}

\cvevent{Chef de projet scientifique / Responsable R\&D}{Berger-Levrault}{janv. 2024 -- présent}{Lyon, France}

\textbf{Périmètre :} programmes R\&D à l’interface industrie / recherche, centrés sur l’IA pour le génie logiciel, la mesure, et la modernisation.

\begin{itemize}
\item \textbf{Études productivité développeur :} définition des protocoles, analyses qualitatives et quantitatives, restitution actionnable aux équipes d’ingénierie.
\item \textbf{IA pour le code :} intégration de capacités IA dans les merge requests et les IDE ; exploration des patterns MCP / tool-calling ; évaluation d’impact en environnement de production.
\item \textbf{GreenIT / numérique responsable :} mesure de la consommation énergétique de l’infrastructure et de l’application/code ; appui à l’éco-conception.
\item \textbf{Cartographie logicielle \& analyse d’architecture :} modélisation et cartographie pour comprendre et moderniser des applications à très grande échelle.
\end{itemize}

Management d’une \textbf{équipe de 6} ingénieurs et chercheurs, et encadrement de plusieurs doctorants.

Encadrement de thèses sur : \textbf{sélection de tests}, \textbf{génération de tests}, \textbf{architecture microservices}, \textbf{localisation de code métier}.

Définition de la feuille de route long-terme et coordination des collaborations avec \textbf{plusieurs laboratoires (Inria)} et unités industrielles.

\divider

\cvevent{Ingénieur logiciel R\&D}{Berger-Levrault}{nov. 2021 -- déc. 2023}{Lyon, France}
\begin{itemize}
  \item Modernisation de front-ends legacy vers \textbf{Angular} (projet Casino ; approche IDM/MDE).
  \item \textbf{Analyse d’architecture} et \textbf{qualité de code} pour orienter les décisions de modernisation.
  \item Encadrement de doctorants et stagiaires ; publications et transfert technologique.
  \item Suivi et animation de partenariats académiques.
\end{itemize}

À l’issue de la thèse, Berger-Levrault et Inria ont signé un accord de partenariat :
\href{https://www.inria.fr/fr/berger-levrault-inria-partenariat-numerique-responsable}{post Inria}.
J’ai poursuivi chez Berger-Levrault afin de prolonger les travaux de modernisation, accompagner le partenariat et lancer des initiatives internes d’analyse logicielle avec la plateforme Moose.

\divider

\cvevent{Ingénieur logiciel doctorant (doctorat industriel)}{Berger-Levrault / Inria RMoD}{janv. 2019 -- oct. 2021}{Montpellier \& Lille, France}
\begin{itemize}
  \item Migration d’applications Java (\textbf{GWT} et \textbf{Swing}, \textbf{1M+ LOC}) vers \textbf{Angular}.
  \item Mise en place d’une architecture hybride (GWT + Angular) pour une migration incrémentale.
  \item Publications et enseignement.
  \item Blog sur Moose : \url{https://modularmoose.org/blog/authors/benoit-verhaeghe/}
\end{itemize}

\divider

\cvevent{Ingénieur logiciel R\&D}{Berger-Levrault}{mars 2018 -- déc. 2018}{Montpellier, France}
\begin{itemize}
  \item Création d’outils de migration d’applications GWT vers Angular.
  \item Méta-modélisation, analyse de code, et état de l’art.
\end{itemize}

\divider

\cvevent{Stage de recherche}{Inria RMoD}{mai -- sept. 2017}{Villeneuve d’Ascq, France}
\begin{itemize}
  \item Apprentissage de Pharo et de l’écosystème ; étude des approches de sélection de tests.
  \item Développement d’un outil de sélection automatique de tests et validation avec la communauté Pharo.
  \item Contribution à l’intégration dans le core Pharo ; rédaction de papiers de recherche.
\end{itemize}

\divider

\cvevent{Développeur logiciel}{University of Emden (Département Informatique)}{avr. -- juil. 2015}{Emden, Allemagne}
\begin{itemize}
  \item Développement d’une application Java utilisant le protocole OPC-UA.
\end{itemize}

\cvsection{Projets}

\cvevent{Extension VS Code --- Support du langage Pharo}{Open-source (projet personnel)}{déc. 2020 -- présent}{}
\begin{itemize}
  \item Écriture et débogage de code Pharo dans VS Code ; usage orienté notebooks.
  \item Fonctionnalités d’assistance par IA intégrées dans l’écosystème.
  \item \textbf{4\,000+} installations sur la marketplace VS Code.
\end{itemize}

\divider

\cvevent{Casino --- Plateforme de migration GUI}{Inria RMoD / Berger-Levrault}{mars 2018 -- déc. 2023}{}
\begin{itemize}
  \item Migration automatique d’IHM applicatives.
  \item Livraison de \textbf{6 importeurs} et \textbf{5 exporteurs}.
\end{itemize}

\divider

\cvevent{Pharo LibVLC}{Open-source (projet personnel)}{}{}
\href{https://github.com/badetitou/Pharo-LibVLC}{Pharo LibVLC} : binding FFI de la bibliothèque VLC pour Pharo.

\divider

\cvevent{SmartTest}{Inria RMoD}{juil. 2017 -- juil. 2018}{Lille, France}
\begin{itemize}
  \item Sélection automatique de tests dans Pharo.
  \item Développement de différentes stratégies d’exécution de tests.
\end{itemize}

\cvsection{Enseignement}

\cvevent{Architecture logicielle}{Polytech Lille}{2019 -- 2021}{France}

\divider

\cvevent{Évolution logicielle \& méta-modélisation}{Université de Montpellier}{2019 -- 2021}{Montpellier, France}

\divider

\cvevent{Algorithmique de base}{Université de Montpellier}{2019 -- 2021}{Montpellier, France}

\cvsection{Formation}

\cvevent{Doctorat en génie logiciel}{Université de Lille / Berger-Levrault}{janv. 2019 -- oct. 2021}{}
Approche incrémentale de migration d’IHM applicatives via des métamodèles

\textbf{Encadrants académiques :}\\
Nicolas Anquetil -- Maître de conférences -- Université de Lille\\
Anne Etien -- Professeure des universités -- Université de Lille\\
\textbf{Encadrant industriel :} Abderrahmane Seriai -- Berger-Levrault

\textbf{Rapporteurs :}\\
Salah Sadou -- Professeur des universités -- Université Bretagne Sud\\
Jean-Rémy Falleri -- Maître de conférences -- Université de Bordeaux

\textbf{Président :}\\
Franck Barbier -- Professeur des universités -- Université de Pau et des Pays de l’Adour

\divider

\cvevent{Diplôme d’ingénieur (Génie logiciel \& Statistiques)}{Polytech Lille}{sept. 2015 -- juil. 2018}{}
\textbf{Encadrants :} Anne Etien, Nicolas Anquetil, Laurent Deruelle, Abderrahmane Seriai

\divider

\cvevent{DUT Informatique}{IUT A -- Université de Lille}{sept. 2014 -- juin 2015}{}

\cvsection{Publications}

\nocite{*}

\printbibliography[heading=pubtype,title={\printinfo{\faFile*[regular]}{Articles de revue}},type=article]

\divider

\printbibliography[heading=pubtype,title={\printinfo{\faUsers}{Actes de conférence}},type=inproceedings]

\divider

\printbibliography[heading=pubtype,title={\printinfo{\faUsers}{Thèse}},type=thesis]

\fi % fin toggle version longue

\end{document}