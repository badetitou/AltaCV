
%%%%%%%%%%%%%%%%%
% This is an sample CV template created using altacv.cls
% (v1.6, 21 May 2021) written by LianTze Lim (liantze@gmail.com). Now compiles with pdfLaTeX, XeLaTeX and LuaLaTeX.
%
%% It may be distributed and/or modified under the
%% conditions of the LaTeX Project Public License, either version 1.3
%% of this license or (at your option) any later version.
%% The latest version of this license is in
%%    http://www.latex-project.org/lppl.txt
%% and version 1.3 or later is part of all distributions of LaTeX
%% version 2003/12/01 or later.
%%%%%%%%%%%%%%%%

%% Use the "normalphoto" option if you want a normal photo instead of cropped to a circle
% \documentclass[10pt,a4paper,normalphoto]{altacv}

\documentclass[10pt,a4paper,ragged2e,withhyper]{altacv}
%% AltaCV uses the fontawesome5 and packages.
%% See http://texdoc.net/pkg/fontawesome5 for full list of symbols.

% Change the page layout if you need to
\geometry{left=1.25cm,right=1.25cm,top=1.5cm,bottom=1.5cm,columnsep=1.2cm}

% The paracol package lets you typeset columns of text in parallel
\usepackage{paracol}
\usepackage[utf8]{inputenc}

% Change the font if you want to, depending on whether
% you're using pdflatex or xelatex/lualatex
\ifxetexorluatex
  % If using xelatex or lualatex:
  \setmainfont{Roboto Slab}
  \setsansfont{Lato}
  \renewcommand{\familydefault}{\sfdefault}
\else
  % If using pdflatex:
  \usepackage[rm]{roboto}
  \usepackage[defaultsans]{lato}
  % \usepackage{sourcesanspro}
  \renewcommand{\familydefault}{\sfdefault}
\fi

% Change the colours if you want to
\definecolor{SlateGrey}{HTML}{2E2E2E}
\definecolor{LightGrey}{HTML}{666666}
\definecolor{DarkPastelRed}{HTML}{450808}
\definecolor{PastelRed}{HTML}{8F0D0D}
\definecolor{GoldenEarth}{HTML}{E7D192}
\colorlet{name}{black}
\colorlet{tagline}{PastelRed}
\colorlet{heading}{DarkPastelRed}
\colorlet{headingrule}{GoldenEarth}
\colorlet{subheading}{PastelRed}
\colorlet{accent}{PastelRed}
\colorlet{emphasis}{SlateGrey}
\colorlet{body}{LightGrey}

% Change some fonts, if necessary
\renewcommand{\namefont}{\Huge\rmfamily\bfseries}
\renewcommand{\personalinfofont}{\footnotesize}
\renewcommand{\cvsectionfont}{\LARGE\rmfamily\bfseries}
\renewcommand{\cvsubsectionfont}{\large\bfseries}


% Change the bullets for itemize and rating marker
% for \cvskill if you want to
\renewcommand{\itemmarker}{{\small\textbullet}}
\renewcommand{\ratingmarker}{\faCircle}

%% Use (and optionally edit if necessary) this .cfg if you
%% want to use an author-year reference style like APA(6)
%% for your publication list
% When using APA6 if you need more author names to be listed
% because you're e.g. the 12th author, add apamaxprtauth=12
\usepackage[backend=biber,style=apa6,sorting=ydnt]{biblatex}
\defbibheading{pubtype}{\cvsubsection{#1}}
\renewcommand{\bibsetup}{\vspace*{-\baselineskip}}
\AtEveryBibitem{\makebox[\bibhang][l]{\itemmarker}}
\setlength{\bibitemsep}{0.25\baselineskip}
\setlength{\bibhang}{1.25em}


%% Use (and optionally edit if necessary) this .cfg if you
%% want an originally numerical reference style like IEEE
%% for your publication list
% \usepackage[backend=biber,style=ieee,sorting=ydnt]{biblatex}
%% For removing numbering entirely when using a numeric style
\setlength{\bibhang}{1.25em}
\DeclareFieldFormat{labelnumberwidth}{\makebox[\bibhang][l]{\itemmarker}}
\setlength{\biblabelsep}{0pt}
\defbibheading{pubtype}{\cvsubsection{#1}}
\renewcommand{\bibsetup}{\vspace*{-\baselineskip}}


%% sample.bib contains your publications
\addbibresource{sample.bib}

\begin{document}
\name{Beno{\^\i}t Verhaeghe}
\tagline{Docteur en Génie Logiciel}
%% You can add multiple photos on the left or right
\photoR{2.8cm}{profil}

\personalinfo{%
  % Not all of these are required!
  \email{work@badetitou.fr}
  \phone{+33 6 58 33 53 74}
  \mailaddress{34bis, avenue Jean Jaurès}
  \location{38600 Fontaine, FRANCE}
  \homepage{badetitou.fr}
  \twitter{@badetitou}
  \linkedin{benoitverhaeghe}
  \github{badetitou}
  \orcid{0000-0002-4588-2698}
}

\makecvheader
%% Depending on your tastes, you may want to make fonts of itemize environments slightly smaller
% \AtBeginEnvironment{itemize}{\small}

%% Set the left/right column width ratio to 6:4.
% \columnratio{0.6}

% Start a 2-column paracol. Both the left and right columns will automatically
% break across pages if things get too long.
% \begin{paracol}{2}
\cvsection{Experiences}

\cvevent{Ph.D. -- Ingénieur R\&D}{Berger-Levrault}{Novembre 2021}{Lyon, FRANCE}
\begin{itemize}
\item Migration d'applications
\item Encadrement de doctorants et de stagiaires
\item Écriture de papiers de recherche
\item Suivi du contrat cadre signé entre Berger-Levrault et Inria
\end{itemize}

Suite à ma thèse, l'entreprise Berger-Levrault et l'Inria ont signé un contrat cadre (\href{https://www.inria.fr/fr/berger-levrault-inria-partenariat-numerique-responsable}{\color{blue}\underline{post inria}}).
Je suis donc resté en poste chez Berger-Levrault en tant qu'ingénieur R\&D afin de poursuivre les travaux de migration de logiciel, et de suivre les doctorants qui découlent de ce contrat cadre.

\divider

\cvevent{Doctorant Génie Logiciel}{Berger-Levrault/Inria RMoD}{Janvier 2019 -- Octobre 2021}{Montpellier/Lille, FRANCE}
\begin{itemize}
\item Migration d'applications GWT vers Angular
\item Mise en place d'architecture hybride
\item Écriture de papiers de recherche
\item Enseignement théorie des modèles
\end{itemize}

\divider

\cvevent{Ingénieur Logiciel}{Berger-Levrault}{Mars 2018 -- Décembre 2018}{Montpellier, FRANCE}
\begin{itemize}
\item Analyse de l’architecture des applications de Berger-Levrault
\item Développement d'outils pour la migration de GWT vers Angular
\end{itemize}

\divider

\cvevent{Etudiant-Chercheur}{RMoD - Inria Lille Nord Europe}{Mai -- Septembre 2017}{Villeneuve d'Ascq, France}
\begin{itemize}
  \item Développement en Pharo d'un "spy" (TestsUsageAnalyser)
  \item Ecriture d'un article scientifique pour l'IWST
  \item Développement d'un outil de sélection de tests (SmartTest)
\end{itemize}

\divider

\cvevent{Developpeur Informatique}{University of Emden,  Departement Computer Science}{Avril -- Juillet 2015}{Emden, Allemagne}
\begin{itemize}
  \item Développement de l'application "Factory Interface"
\end{itemize}


\cvsection{Projets}

\cvevent{Pharo Language Server}{Inria RMoD}{Décembre 2021}{Grenoble, FRANCE}
\begin{itemize}
\item Connecter Pharo avec VSCode
\item +21 étoiles GitHub
\end{itemize}

\divider

\cvevent{Casino}{Inria RMoD}{Février 2018}{Lille, FRANCE}

Casino est l'outil que j'ai développé pendant ma thèse et dont l'objectif est de faciliter la migration des front-ends d'application.
L'outil est composé d'un méta-modèle, de plusieurs importeurs et plusieurs exporteurs.
Il est disponible sur GitHub.

\divider

\cvevent{SmartTest}{Inria RMoD}{Juillet 2017 -- 2019}{Lille, FRANCE}
\begin{itemize}
\item Sélection de tests automatiques (Pharo)
\item Développement de différentes stratégies d’exécution de test (Pharo)
\end{itemize}

\divider

\cvevent{TestsUsageAnalyser}{Inria RMoD}{Avril -- Aout 2017}{Lille, FRANCE}
\begin{itemize}
  \item Espion collecteur de données
\end{itemize}
  

% \medskip

% \cvsection{A Day of My Life}

% % Adapted from @Jake's answer from http://tex.stackexchange.com/a/82729/226
% % \wheelchart{outer radius}{inner radius}{
% % comma-separated list of value/text width/color/detail}
% \wheelchart{1.5cm}{0.5cm}{%
%   6/8em/accent!30/{Sleep,\\beautiful sleep},
%   3/8em/accent!40/Hopeful novelist by night,
%   8/8em/accent!60/Daytime job,
%   2/10em/accent/Sports and relaxation,
%   5/6em/accent!20/Spending time with family
% }

% use ONLY \newpage if you want to force a page break for
% ONLY the current column


%% Switch to the right column. This will now automatically move to the second
%% page if the content is too long.
% \switchcolumn

% \cvsection{My Life Philosophy}

% \begin{quote}
% ``Something smart or heartfelt, preferably in one sentence.''
% \end{quote}

\cvsection{Enseignements}

\cvevent{Architectures Logicielles}{Polytech'Lille}{2019 -- 2021}{Lille, FRANCE}

\begin{itemize}
\item \faClock 75 heures
\item 2ème année d'école d'ingénieur (Master 1)
\end{itemize}

J'ai encadré pendant trois ans des étudiants pendant les TPs d'architecture logicielles de Polytech'Lille.
Puis j'ai suivi des groupes d'étudiants pendant leurs projets d'architectures logicielles.
L'objectif de cet enseignement était de faire découvrir les architectures 3-tiers.

\divider

\cvevent{Structures de données}{Polytech'Lille}{2019}{Lille, FRANCE}
\begin{itemize}
\item \faClock 12 heures
\item 1er année d'école d'ingénieur (Licence 3)
\end{itemize}

J'ai enseigné les TPs de structures de données à Polytech'Lille aux étudiants de première années du cycle ingénieur.
L'objectif était de faire développer et manipuler les structures de données en C.
En cas d'exemple, nous avons aussi demandé aux étudiants d'utiliser les structures de données afin de résoudre des problèmes venus de la thèorie des graphes.

\divider

\cvevent{Système d'exploitation}{Universit\'e de Montpellier}{2020}{Montpellier, FRANCE}

\begin{itemize}
  \item \faClock 40 heures
  \item Licence 2
\end{itemize}

J'ai suivi une trentaine d'étudiants pendant dix séances TPs de 4 heures pour les accompagner dans les apprentissages liés au système d'exploitation.
Ces enseignements visés à enseigner les languages de programmation C et Python.
En particulier, l'objectif était de faire comprendre le fonctionnement de la compilation séparée,
les différentes structures de données
et du fonctionnement du système d'entrées et sorties d'une application.

\divider

\cvevent{Evolution et restructuration de logiciels}{Universit\'e de Montpellier}{2019 -- 2020}{Montpellier, FRANCE}
\begin{itemize}
\item \faClock 16h
\item Master 2
\end{itemize}

J'ai enseigné durant ce cour à des étudiants de Master 2 de l'université de Montpellier.
Pour cet enseignement, j'ai du concevoir, le cours, le TP et le TD.
Les étudiants ont pu découvrir le language de programmation Pharo, la platforme d'analyse Moose et l'utilisation des modèles et méta-modèles afin d'extraire de l'information à partir du code d'un logiciel qu'ils ne connaissaient pas.

\divider

\cvevent{Fondamentaux architectures et systèmes d’exploitatio}{Polytech'Montpellier}{2020}{Montpellier, FRANCE}
\begin{itemize}
\item \faClock 2h
\item 1er année d'école d'ingénieur (Licence 3)
\end{itemize}

J'ai pu donner un cours sur le fonctionnement des processus sur Linux.
En particulier, leurs fonctionnement, comment utiliser les \textit{fork}, lister les processus, et envoyer des signaux aux processus an utilisant Python.

\divider

\cvevent{Architectures et Systèmes appliqués à l’IOT}{Polytech'Montpellier}{2020}{Montpellier, FRANCE}
\begin{itemize}
  \item \faClock 17h
  \item 1er année d'école d'ingénieur (Licence 3)
\end{itemize}

Durant cet enseignement j'ai suivi des étudiants pendant leurs découvertes de la programmation Python pour controler un Arduino.
J'ai dans un premier temps fait l'enseignement des TPs, puis le suivi de projet et enfin l'évaluation des projets.

\cvsection{Succès}

\cvachievement{\faTrophy}{3ème place Innovation Award}{Pharo Language Server est arrivé troisième à l’Innovation Award de ESUG2022}

\divider


\cvachievement{\faTrophy}{Best Paper Award}{Migrating GWT to Angular 6 using MDE @ Sattose}

\divider

\cvachievement{\faTrophy}{2ème place Innovation Award}{SmartTest est arrivé deuxième à l’Innovation Award de ESUG2017}

\divider

\cvachievement{\faTrophy}{2ème place Nuit de l’Info}{J’étais responsable d’une équipe durant la Nuit de l’Info 2016}

% \divider

% \cvachievement{\faHeartbeat}{Another achievement}{more details about it of course}

\cvsection{Compétences}

\cvtag{Pharo}
\cvtag{Java/JEE}
\cvtag{Angular}
\cvtag{GWT}
\cvtag{Spring}
\cvtag{C}

\cvsection{Langues}

\cvskill{Français}{5}
\divider

\cvskill{Anglais}{4}

%% Yeah I didn't spend too much time making all the
%% spacing consistent... sorry. Use \smallskip, \medskip,
%% \bigskip, \vspace etc to make adjustments.
\medskip

\cvsection{Formation}

\cvevent{Ph.D. en Génie Logiciel}{Université de Lille / Berger-Levrault}{Janvier 2019 -- Octobre 2021}{}
Support à la migration d'applications GWT vers Angular

\divider

\cvevent{Ingénieur Informatique et Statistique}{Polytech Lille}{Septembre 2015 -- Juillet 2018}{}

\divider

\cvevent{DUT Informatique}{IUT A - Université de Lille}{Sept 2014 -- Juin 2015}{}

% \divider

\cvsection{Hobbies}

\cvevent{Le Club Info -- Président}{Polytech Lille}{2016 -- 2018}{}

\divider

\cvevent{Tennis de table -- Trésorier}{Lezennes, FRANCE}{2016 -- 2018}{}

\divider

\cvevent{Recherche Historique -- Secrétaire}{CRHL Lezennes, FRANCE}{2014 -- février 2018}{}


% \cvsection{Referees}

% % \cvref{name}{email}{mailing address}
% \cvref{Prof.\ Alpha Beta}{Institute}{a.beta@university.edu}
% {Address Line 1\\Address line 2}

% \divider

% \cvref{Prof.\ Gamma Delta}{Institute}{g.delta@university.edu}
% {Address Line 1\\Address line 2}


% \end{paracol}

\cvsection{Publications}

\nocite{*}

% \printbibliography[heading=pubtype,title={\printinfo{\faBook}{Books}},type=book]

% \divider

\printbibliography[heading=pubtype,title={\printinfo{\faFile*[regular]}{Journal Articles}},type=article]

\divider

\printbibliography[heading=pubtype,title={\printinfo{\faUsers}{Conference Proceedings}},type=inproceedings]

\divider

\printbibliography[heading=pubtype,title={\printinfo{\faUsers}{PhD thesis}},type=thesis]


\end{document}
